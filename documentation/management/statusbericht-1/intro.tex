\setcounter{chapter}{1}
\section{Status}
Ein Kickoff-Treffen und ein Kickoff-Treffen mit dem Tutor wurden abgehalten.
Es wurde auf Projekt und Rolleneinteilung geeinigt.

Das Projekt wurde in Untergruppen geteilt, die teilweise schon mit der Arbeit 
begonnen haben: 

\subsection{GUI}
Peter und Chris beschäftigen sich mit der GUI, genauer gesagt mit der Analyse 
der Workflows/typischen Arbeitsszenarien, sowie später der Analyse der 
derzeitigen Umsetzung. 

Das Dokument Workflows wird in Woche 45 erwartet, das Dokument Analyse in 
Woche 46. Darauffolgend soll später ein Dokument der Implementierungsvorschläge entstehen.

\subsection{Codereview}
Dominik und Christopher analysieren den derzeitigen Codebestand auf 
strukturelle Mängel. Das Codereviewdokument, das aus Problemen, 
Behebungsvorschlägen und Ideen für neue Features besteht, ist bereits zur Hälfte 
fertiggestellt.

Das Codereviewdokument wird zu Beginn der Woche 45 erwartet.

\subsection{XMPP}
Johannes hat die Library "Muse" für die Implementierung des XMPP-Backends 
ausgewählt. Entscheidend war Reife der Entwicklung, Vollständigkeit der 
Implementierung (XEPs) und Lizenzen. Da die Libraries allerdings sehr ähnliche
Konzepte verfolgen, sollte ein Austauschen, falls zu einem späteren Zeitpunkt 
notwendig, nicht übermäßig aufwändig sein.

Die Implementierung wird agil stattfinden: Entwicklungszyklenweise (etwa 2 
Wochen) wird geplant und implementiert. Die Entwicklung wird mit Woche 45 
beginnen.

\subsection{Qualitätsreview}
Das Review der bestehenden Dokumente wurde von Simon durchgeführt. 
Das zusammenfassende Dokument wird Ende der Woche 44 erwartet.

\section{Statusindikatoren}

\subsection{Zeitplanung}
Keine Verschiebungen notwendig.

\subsection{Inhaltliche Planung}
Keine Verschiebungen notwendig.

\subsection{Entscheidungen}
Projektfindung, Arbeitsaufteilung wurden plangemäß durchgeführt.


