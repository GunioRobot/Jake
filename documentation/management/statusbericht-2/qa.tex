\section{Dokumentation und QA}
Zur Übersicht über die Dokumentation wurde im Wiki ein Documentation Center
eingerichtet\footnote{\texttt{http://sepm.doublesignal.com/wiki/doku.php?id=documentation\_center}}.
Die Dokumentation gliedert sich in drei Bereiche:

\begin{itemize}
\item \emph{Interne Dokumentation}: Kommunikation und Dokumentation innerhalb
des Teams und des Projekts. Diese wird im Wiki in deutscher Sprache verfasst.
\item \emph{konkrete öffentliche Dokumentation}: Dokumentation die die konkrete
Implementierung betrifft und auch nach dem Abschluss von ASE noch von Bedeutung
ist. Konkrete technische Doku, konkrete UI Doku, etc. Dokumentiert in Maven.
\item \emph{allgemeine öffentliche Dokumentation}: Dokumentation die unabhängig
von einer konkreten Implementierung besteht. Dabei handelt es sich um die \emph
{specification}, die auch die use cases und die allgemeine UI methodology
enthält. Dokumentiert in \texttt{specification.pdf}.
\end{itemize}

\subsubsection{QA}
Die Dokumente der QA werde erst im weiteren Verlauf des Projektes festgelegt
und erstellt. Vorgaben zu unit tests und der testing policy werden sich in den
\emph{coding guidelines} in Maven finden, GUI Tests werden mit \emph{mozilla
litmus}\footnote{http://dev.jakeapp.com:8080/tests/} organisiert.

