\setcounter{chapter}{1}
\section{Status}
Es wurde festgestellt, dass Unklarheiten und unterschiedliche Ansichten über 
die Detailziele der Phase herrschen. Daher wurden die Tätigkeiten der 
Gruppen XMPP, GUI und Codereview verschoben und stattdessen an dem Dokument 
SRS gearbeitet (Woche 44-46).

Gleichzeitig (Woche 45-46) wurde von der Qualitätssicherung die 
Rahmenbedingungen und Dokumente der einzelnen Gruppen besprochen (siehe 
Protokolle, etwa Code-Guidelines,).

Es wurden die qualitativen Ziele des Projekts festgelegt.

\subsection{Qualitative Ziele}
Die Arbeitszeit pro Projektmitglied soll nicht 150h überschreiten und 
möglichst ausgeglichen zwischen Teammitgliedern sein. Falls nicht möglich, 
sollen eher unwichtige Zusatzfeatures weggelassen werden als die Stundenzahl 
grob zu überziehen.

Das Ziel ist, das Projekt am Ende des Semesters releasefertig zu haben. Wir 
entscheiden uns, zum Open-Source-Release hinzuarbeiten und nicht bloss zur 
LVA-Abgabe. Das Produkt soll gut genug sein um es anderen empfehlen zu 
können und der Öffentlichkeit zu präsentieren.

Wann und unter welcher Lizenz das Release stattfindet ist dabei zweitrangig 
und noch offen.

Die Performance des Programms soll möglichst Ressourcen-schonend sein.

\subsection{Software Requirements Specification (SRS)}
Die SRS spezifiziert exakt, was der Benutzer von dem fertigen Produkt Jake zu 
erwarten hat und macht die Vorstellungen und Detailziele der Phase explizit.

Das Dokument ist beigelegt.

\section{Derzeitiger Fortschritt}
\subsection{GUI}
Auf Basis der SRS werden die Use cases von QA und GUI-Team erstellt. Des 
weiteren werden die Programm-Workflows vom GUI-Team beschrieben (wie das 
Programm bedient werden soll). 
Aus den Workflows soll später der GUI-Entwurf/Prototyp vom GUI-Team erstellt
 werden.
\subsection{Technische Architektur}
Schnittstellen zwischen \emph{GUI} und \emph{core}-Komponente und einige andere müssen 
aufgrund der SRS überdacht und designed werden. Des weiteren werden zusätzliche,
neue Features (siehe SRS), etwa eine Liste zuletzt geöffneter Projekte, 
entworfen werden.
\section{Planung}
Die Use-Cases mit Workflows sollten am 2.12. abgeschlossen sein. Der 
GUI-Entwurf/Prototyp wird zum MR2 fertig.

Ebenso soll beim MR2 ein Prototyp des XMPP-Backends auf Basis des alten 
Programms fertig sein.

Die Technische Architektur wird zum MR2 die neuen Komponenten/Klassendiagramme 
und Interfacedefinitionen abgeben.

\section{Statusindikatoren}

\subsection{Zeitplanung}
Die geplante Programmierung des XMPP-Backends hat sich verzögert und wird erst 
mit 26.11. beginnen.

\subsection{Inhaltliche Planung}
Die SRS aktualisiert die Planung aus dem Ausgangsprojekt.

\subsection{Entscheidungen}
Beschluss der SRS, qualitative Ziele.

















