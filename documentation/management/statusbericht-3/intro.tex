\setcounter{chapter}{1}
\section{Status}
Das Projekt Jake ist nach 8 Sprints in die Endphase eingetreten. Die Entwicklung einiger Komponenten sind stark verzögert:

\subsection{Hibernate}
Die (Neu-)Entwicklung der DAOs und die Formulierung der Mappings nahm unglaublich viel Zeit in Anspruch, sodass die DAOs erst am 16.1.2009 fertiggestellt werden konnten.

\subsection{ICS/ICS-XMPP}
Diese Komponenten wurden fertiggestellt. Sie unterstützen nun Kommunikation und Statusabfragen innerhalb einer Gruppe von Mitgliedern, was in XMPP durch eine Gruppe am Roster abgebildet wird. Des weiteren wurden Transfermöglichkeiten für große Inhalte implementiert, die zuerst direkte Client-zu-Client Verbindung über Sockets verwendet, und, falls dies nicht funktioniert, auf XMPP In-Band File-Transfers zurückfällt.

\subsection{GUI/Core}
Zahlreiche Dialoge und vierzehn der Use Cases wurden fertiggestellt. Bei fast allen Use Cases fehlt noch die ``reale'' Implementierung im Core.

\subsection{Zeitplanung}
Die Entwicklung ist verzögert und es werden dringen Maßnahmen zur Fertigstellung benötigt.

\subsection{Inhaltliche Planung}
Feature-Freeze wurde am 1.1. 2009 festgelegt.

\section{Entscheidungen}
Um die massive Verzögerung und die fehlende Implementierung nachzuholen wird am Wochenende 17.1.-18.1. ein ``Hackfestival'' ein. Die Mitglieder treffen sich zu einer intensiven, hoffentlich produktiven, Programmiersession.









