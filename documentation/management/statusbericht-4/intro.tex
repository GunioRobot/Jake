\setcounter{chapter}{1}
\section{Status}
Das Projekt Jake konnte nach verlängerter Entwicklungszeit fertiggestellt werden.

\section{Komponenten}
An den Komponenten \emph{FSS, ICS, ICS-XMPP} mussten lediglich kleine Änderungen und Bugfixes vorgenommen werden. 
Die großen Arbeitspakete seit dem letzten Bericht betrafen \emph{Core} und \emph{GUI}.

\subsection{Domainobjects, Datenbank}
Sämtliche Domainobjekte wurden von Johannes im Februar reviewed und überarbeitet um eine darauf aufbauende Weiterentwicklung zu ermöglichen. Ebenso wurden die DAOs getestet und fertigimplementiert. 

\subsection{JakeCommander}
Zum Testen und für einen Servermodus wurde eine durch Kommandozeile ansprechbare Implementierung geschrieben, die die meisten Use-Cases abdeckt.


\subsection{Core}
Die Use-Cases wurden fertigimplementiert, und die Übertragungs- und Synchronisationslogik konnte fertiggestellt werden.


\section{GUI}
Am GUI mussten keine Konzepte geändert werden, lediglich die Implementierung der Use-Cases wurde vervollständigt und poliert.


\section{Neuerungen}
Noch zu notieren wäre ein interessantes Hibernate-Problem, dass lediglich eine Hibernate-Datenbankverbindung an einen Thread gebunden werden kann. Daher wurden alle Datenbankaufrufe in einen Dispatcherthread ausgelagert, der alle Datenbankanfragen pro Datenbank ausführt.

Zur Verbesserung der Performance wurde ein Cachingsystem entworfen, dass in verschiedenen Granularitäten Daten gecacht und aktualisiert werden können.


\section{Abschließende Bemerkungen}
Das Projekt Jake erreichte mit etwa 60.000 Codezeilen nach über 1500 Arbeitsstunden mit den ausgereiften Projekten Checkstyle, Cobertura oder Log4J vergleichbare Größe.

Obwohl uns das Projekt Strapazen abverlangte, sind wir doch mit dem produzierten Ergebnis zufrieden.

\section{Anhänge}
\begin{itemize}
\item Gantt-Diagramm (ab Seite 5)
\item Klassendiagramm (ab Seite 8)
\item Datenbankbeschreibungen mit ER-Diagrammen (ab Seite 12)
\item Stundenlisten geführt bis 30.1. (ab Seite 18)
\end{itemize}

Die Dokumente UI-Skizzen und -beschreibung, Risikoabschätzung, 
Anwendungsfallbeschreibung, Komponentendiagramm und Domänenmodell wurden bereits 
früher abgegeben und sind in der vorliegenden Version noch zutreffend.

Die Besprechungsprotokolle sind auf http://wiki.jakeapp.com/doku.php?id=protokolle:protokolle zu finden. 

Die Javadoc (Technische Dokumentation) ist auf http://wiki.jakeapp.com/doku.php?id=javadoc zu finden. 

