% Dieser Teil korrespondiert mit der Projektbeschreibung. 
% Er legt fest was unser Projekt nicht ist, und was daher auch nicht implementiert wird.

\section{Projektabgrenzung}
\begin{itemize}
  \item Im Rahmen der Laborüubung SEPM wird nur die erste Phase laut Projektbeschreibung implementiert. Die konkrete Implementierung der Netzwerk- und Synchronisationsfunktionalität erfolgt erst in einer späteren Projektphase und wird in der ersten Phase durch Mock-Objekte simuliert.
  \item Es ist nicht möglich, in Echtzeit gleichzeitig an einem Dokument zu arbeiten (wie etwa in Gobby oder Google Apps).
  \item Es soll kein SCM oder Versionsmanagement implementiert werden. Alte Versionen von Dateien werden nicht behalten und sind daher auch nicht wiederherstellbar.
  \item Da der Fokus auf Nutzungsumgebungen liegt, die primär binäre beziehungweise proprietäre Formate verwenden, soll kein automatisches Mergen (etwa von Textdateien) implementiert werden.
  \item Sämtliche Synchronisation soll nur zwischen Clients ablaufen und es soll keinen Server geben, der Projektdaten zentral verwaltet oder die Problematik von "verlorenen" Dateien beim Offlinegehen des letzten Clients mitigiert. Ein ähnliches Verhalten kann aber durch einen ständig online befindlichen Client im "passive mode" emuliert werden.
  \item Es soll kein Editor für Dateien direkt in die Applikation integriert sein. Das Bearbeiten von Dateien wird mittels externer Anwendungen gehandhabt.
  \item Es gibt keine explizite Rechteverwaltung, in der User Rechte für die von ihnen eingestellten Dateien festlegen, d.h. alle User sind gleichberechtigt.
  \item Es ist nicht möglich, über mehrere Ordner verteilte Dateien zu sharen, d.h. es gibt einen einzigen Projektordner, dessen Inhalt samt Unterordnern synchronisiert wird und es ist nicht möglich im Projektordner Dateien auszuwählen, die nicht synchronisiert werden sollen.
\end{itemize}