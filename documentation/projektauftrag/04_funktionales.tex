% funktionale Anforderungen, Anwendungsfälle
%    Eine Liste von funktionalen Anforderungen oder eine Liste von User-Level Use Cases. Dabei sind nicht die ausführlichen Use Cases selbst gemeint (die kommen erst später) sondern lediglich gut beschreibende one-liner, eine Weiterführung der Featureliste. z.B. Datei zum Projektpool hinzufügen, neues Projekt erstellen, neues Projektmitglied hinzufügen

\section{Funktionale Anforderungen}
% Es soll entweder eine Liste von Anforderungen oder eine Liste von (User-Level)
% Anwendungsfällen erstellt werden. Dabei soll die Featureliste aus dem
% Projektvorschlag hilfreich sein. Egal ob Anwendungsfälle oder Anforderungen, beide
% können Gewichtet und aufgeteilt werden um eine Entscheidungsgrundlage für das
% Arbeitsprogramm und die WBS zu erstellen. (z.B. Need-to-Have, Nice-to-Have)
% BEISPIEL
% Anforderungen: User Interface soll Eingabemasken zur Verwaltung von Studenten
% und Prüfungen implementieren. Studentenlisten sollen über einen grafischen
% Dateiauswahl- bzw. Dateispeicherdialog als XML geöffnet und gespeichert werden
% können.
% Anwendungsfälle: Studenten verwaten/exportieren, Prüfung anmelden/absolvieren

% Hier bitte ENTWEDER alles Anwendungsfälle ODER alles Features schreiben
% nicht mischen!

\begin{itemize}
  \item Authentifizierung bei Programmstart mittels User/Passwort gegen das Netzwerkinterface
  \item Erstellen eines Projektes mit Auswahl eines Projektordners.
  \item Bearbeiten von Benutzern eines Projektes (mittels Netzwerkservice)
  \item Teilnahme an einem Projekt bestätigen/ablehnen.
  \item Erstellen/Verwalten von Labels/Tags für Datenobjekte. (Dateien, Notizen)
  \item Zuweisen/Entfernen von Labels zu einer/mehreren Dateien.
  \item Dateiverwaltung (hinzufügen, entfernen) von Dateien/Ordner.
  \item Hinzufügen von Metadaten zu Datenobjekten.
  \item Erstellen von Notizen.
  \item Suchen von Dateien mittels Name, Label/Tag
  \item Verbindungsmanagement zum Netzwerk. (verbinden, trennen)
  \item Daten/Notizen können gepusht/gepullt werden.
  \item Feature Auto-pull: Änderungen werden automatisch heruntergeladen.
  \item Feature Auto-push: Änderungen werden automatisch verteilt.
\end{itemize}