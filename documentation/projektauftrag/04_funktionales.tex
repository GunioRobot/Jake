% funktionale Anforderungen, Anwendungsfälle
%    Eine Liste von funktionalen Anforderungen oder eine Liste von User-Level Use Cases. Dabei sind nicht die ausführlichen Use Cases selbst gemeint (die kommen erst später) sondern lediglich gut beschreibende one-liner, eine Weiterführung der Featureliste. z.B. Datei zum Projektpool hinzufügen, neues Projekt erstellen, neues Projektmitglied hinzufügen

\section{Funktionale Anforderungen}
\begin{itemize}
  \item Authentifizierung bei Programmstart mittels User/Passwort (Login gegen das Netzwerkinterface)
  \item Erstellen eines Projektes mit Auswahl eines Projektordners.
  \item Bearbeiten von Benutzern eines Projektes.(mittels Netzwerkservice)
  \item Teilnahme an einem Projekt bestätigen/ablehnen.
  \item Erstellen/Verwalten von Labels/Tags für Datenobjekte. (Dateien, Notizen)
  \item Zuweisen/Entfernen von Labels zu einer/mehreren Dateien.
  \item Dateiverwaltung (hinzufügen, entfernen) von Dateien/Ordner.
  \item Hinzufügen von Metadaten zu Datenobjekten.
  \item Erstellen von Notizen.
  \item Suchen von Dateien mittels Name, Label/Tag
  \item Verbindungsmanagement zum Netzwerk. (verbinden, trennen)
  \item Daten/Notizen können gepusht/gepullt werden.
  \item Feature Auto-pull: Änderungen werden automatisch heruntergeladen.
  \item Feature Auto-push: Änderungen werden automatisch verteilt.
\end{itemize}