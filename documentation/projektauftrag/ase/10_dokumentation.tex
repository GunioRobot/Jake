%Informationswesen/Dokumentation
%   Interne Kommunikation, Dokumentation, Dokumentationsrichtlinien, Reviewprozess


\section{Informationswesen/Dokumentation}
\subsection{Interne Kommunikation}
Zur internen Kommunikation wird ein eigens eingerichtetes Wikisystem verwendet.
Ankündigungen und wichtige Mitteilungen werden über eine Mailingliste verteilt. 
Die gesamte Projektgruppe trifft sich etwa einmal pro Woche zu einem ein- bis eineinhalbstündigen Meeting. 
Die Agenda für die Meetings wird zuvor im Wiki bekanntgegeben und kann dort diskutiert werden. 

\subsection{Externe Kommunikation}
Die externe Kommunikation wird während der Übung über regelmäßig stattfindende Review-Meetings sowie über Email-Kommunikation mit unserem Tutor und Assistenten geführt.

\subsection{Organisatorische Dokumentation}
Die organisatorische Dokumentation des Projekts wird über das Wiki abgewickelt. 
Protokolle, Stundenlisten, etc. werden während der Dauer der Übung laufend aktualisiert.

\subsection{Technische Dokumentation}
Die gesamte technische Dokumentation und Spezifikation wird über Maven abgewickelt. 
Dies umfasst Dokumente der technischen Planung, Dokumente der Anforderungsspezifikation, 
Dokumente der Qualitätssicherung, Dokumentation auf Codeebene, Endbenutzer-Dokumente, etc.

Sämtliche technischen Spezifikationen werden in Englisch verfasst.
