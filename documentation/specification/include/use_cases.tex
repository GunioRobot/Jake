\part{Use Cases}
\section{General}

%reset the uc fields
\def\ucreset{
\def\ucid{-}
\def\ucname{-}
\def\ucrationale{-}
\def\ucsummary{-}
\def\ucpreconditions{-}
\def\uctriggers{-}
\def\ucprimaryscenario{-}
\def\ucalternativepath{-}
\def\ucexceptions{-}
\def\ucpostconditions{-}
\def\ucfrequency{-}
\def\ucsee{-}
}

\def\ucprint{
\subsection{\ucid: \ucname}
\begin{tabular}{|p{3.3cm}|p{12cm}|}
\hline
ID & \ucid \\ 
\hline 
Name & \ucname \\
\hline
%Status & \ucstatus \\ 
%\hline 
Goal/Rationale & \ucrationale \\ 
\hline 
Summary & \ucsummary \\ 
\hline
Preconditions & \ucpreconditions \\
\hline
Triggers & \uctriggers \\
\hline
Primary Scenario & \ucprimaryscenario \\
\hline
Alternative Path & \ucalternativepath \\
\hline
Postconditions & \ucpostconditions  \\
\hline
Exceptions & \ucexceptions \\
\hline
Frequency of use & \ucfrequency  \\
\hline
See also & \ucsee \\
\hline
\end{tabular}
}


\ucreset
\def\ucid{GEN-01}
\def\ucname{Launch Jake}
\def\ucrationale{The user wants to start Jake when it is not currently running}
\def\ucsummary{Jake is started and the main window is displayed}
\def\ucpreconditions{A running operating system with Jake installed and all JVM and library prerequisites met}
\def\uctriggers{Launching Jake from the OS}
\def\ucprimaryscenario{
\begin{enumerate}
\item Jake starts and the main window is displayed.
\item Open projects from the previous session (if any) are restored.
\end{enumerate}}
\def\ucpostconditions{Jake is running. If the user has used Jake before and the previous session's login credentials have been saved, Jake automatically kickstarts use case NET-01.}
\def\ucfrequency{Daily/Always}
\ucprint

\section{Network Operations}
\ucreset
\def\ucid{NET-01}
\def\ucname{Log in to Jabber}
\def\ucrationale{The user wants to connect to the network in order to use Jake}
\def\ucsummary{In order to use Jake, a connection to a Jabber network is required. This connection is established by the user entering their network credentials.}
\def\ucpreconditions{Jake is running (GEN-01) and the user is not yet logged in}
\def\uctriggers{User interaction (active) or application launch with saved credentials (passive)}
\def\ucprimaryscenario{
\begin{enumerate}
\item User chooses to log in.
\item The user is prompted to enter username and password for the network.
\item Jake attempts to connect to the server and log in.
\item The user is informed of a successful connection.
\end{enumerate}
}
\def\ucalternativepath{If there are saved credentials from a previous session, Jake uses these to automatically login without explicit user action.}
\def\ucpostconditions{The user is logged in.}
\def\ucexceptions{If the credentials provided in step 2 are incorrect or the Jabber server cannot be reached, the user shall be informed of this and given the option to resume at step 2 in the primary scenario.}
\def\ucfrequency{Daily / Always (on startup)}
\ucprint

\ucreset
\def\ucid{NET-02}
\def\ucname{Log out from Jabber}
\def\ucrationale{In some situations, the user may want to explicitly disconnect from the network (and not just exit the application), e.g. when they want to register a different account or do not want other people using their account to be able to use Jake with their ID. For this reason, they are able to log out, leaving the application in a similar state as after its first run.}
\def\ucsummary{Log out from the currently used network ID}
\def\ucpreconditions{The user is logged in (NET-01)}
\def\uctriggers{User action (active)}
\def\ucprimaryscenario{
\begin{enumerate}
\item The user chooses to log out.
\item The user is logged out from the network.
\end{enumerate}}
\def\ucalternativepath{-}
\def\ucpostconditions{The user is logged out.}
\def\ucfrequency{Monthly / Seldom}
\ucprint

\ucreset
\def\ucid{NET-03}
\def\ucname{Register Jabber ID}
\def\ucrationale{The user wants a Jabber account to be able to use Jake}
\def\ucsummary{Users who do not yet have a Jabber account need to create one before using Jake. Jake guides them through the process so they can create an account on a server from within Jake.}
\def\ucpreconditions{Jake is running (GEN-01)}
\def\uctriggers{User action (active)}
\def\ucprimaryscenario{
\begin{enumerate}
\item The user chooses to register.
\item The user is prompted to pick a server from a preconfigured list or enter one of their own choice.
\item The user is prompted to fill in a form with the required details (username, password).
\item The account is created on the server.
\item The user is automatically logged in with the new account that was just created.
\end{enumerate}
}
\def\ucalternativepath{The user may cancel this operation, returning to NET-01.}
\def\ucpostconditions{The user has a new Jabber account and is connected to the network with it, now able to use Jake.}
\def\ucexceptions{If the credentials provided in step 3 are incorrect (e.g. password to short, username already taken) or the Jabber server selected in step 2 cannot be reached, the user shall be informed of this and given the option to resume at the respective step in the primary scenario.}
\def\ucfrequency{Yearly / Once (usually one jabber id is enough)}
\ucprint


\section{Project Management}

\ucreset
\def\ucid{PRJ-01}
\def\ucname{Create Project}
\def\ucrationale{The user wants to create a new project to share with other users}
\def\ucsummary{Creates a Jake project from a (new or existing) local folder}
\def\ucpreconditions{Jake is running (GEN-01)}
\def\uctriggers{User action (active)}
\def\ucprimaryscenario{
\begin{enumerate}
\item The user chooses to create a new project.
\item The user specifies a local project folder (which may, but doesn't have to, already exist) and a project name. 
\item The project is created and added to the list of active projects.
\item Any files which already exist in the local folder become local files of this project.
\end{enumerate}
}
\def\ucalternativepath{-}
\def\ucpostconditions{The new project has been created and any files already present in the project folder are ready to be shared.}
\def\ucexceptions{
If the chosen directory cannot be accessed (e.g. because of filesystem issues or restrictive permissions) or either the directory itself or a subdirectory of it already is a Jake project, the user is informed of this and given the option to resume at step 2 in the creation process.
}
\def\ucfrequency{Weekly / Often}
\ucprint


\ucreset
\def\ucid{PRJ-02}
\def\ucname{Receive project invitation}
\def\ucrationale{In order for Jake to be useful to the user, they need to be member of at least one project. By receiving a project invitation, they can easily join a project.}
\def\ucsummary{When users receive project invitations, they may accept them, resulting in that project being created locally.}
\def\ucpreconditions{The user is logged in (NET-01).}
\def\uctriggers{An invitation is received (passive)}
\def\ucprimaryscenario{
\begin{enumerate}
\item An invitation to a project from another Jake user is received. 
\item The user is asked to accept/deny the invitation.
\item The user selects a target directory for the new project.
\item The project folder is created.
\item The project is added to the current projects.
\end{enumerate}
}
\def\ucalternativepath{If the user declines the invitation, it may either be discarded completely or remain in a list of pending invitations for accepting later.}
\def\ucexceptions{If the chosen target directory already exists and is NOT empty, the user is warned of the problems this may cause and prompted if they want to proceed.}
\def\ucpostconditions{The project specified by the invitation is now active in the local Jake workspace.}
\def\ucfrequency{Weekly/Often}
\ucprint


\ucreset
\def\ucid{PRJ-03}
\def\ucname{Open project}
\def\ucrationale{The user wants to reopen an existing project that is not currently open in Jake, either because it has been closed manually, or for some other reason (e.g. reinstalled Jake)}
\def\ucsummary{A user may open an existing project by specifying the location of the project folder. The specified project is loaded.}
\def\ucpreconditions{Jake is running (GEN-01) and the project to be opened is not already open.}
\def\uctriggers{User action (active)}
\def\ucprimaryscenario{
\begin{enumerate}
\item The user chooses to open an existing project.
\item The user is asked to select the project folder in the local filesystem.
\item The project is opened.
\end{enumerate}
}
\def\ucalternativepath{If an invalid path is selected or the folder chosen is not a Jake project, the user is given the option to resume at step 2 in the process. Additionally, the application could suggest that the user creates a new project at that path (PRJ-01).}
\def\ucpostconditions{The project is open and ready to use}
\def\ucfrequency{Monthly/Sometimes}
\ucprint


\ucreset
\def\ucid{PRJ-04}
\def\ucname{Close project}
\def\ucrationale{The user wants to close a project, e.g. when it is not needed for a longer period of time.}
\def\ucsummary{A user may close a project. Closed projects are not displayed in Jake.}
\def\ucpreconditions{Jake is running (GEN-01) and the project to be closed is open and selected.}
\def\uctriggers{User action (active)}
\def\ucprimaryscenario{
\begin{enumerate}
\item The user chooses to close the project.
\item Confirmation is requested from the user.
\item The project is closed.
\end{enumerate}
}
\def\ucalternativepath{-}
\def\ucpostconditions{The project is closed, no longer active and no longer visible in Jake.}
\def\ucexceptions{-}
\def\ucfrequency{Monthly/Sometimes}
\ucprint


\ucreset
\def\ucid{PRJ-05}
\def\ucname{Pause project}
\def\ucrationale{The user wants to temporarily suspend a project (e.g. when it is currently not needed)}
\def\ucsummary{The user may pause a project. Paused projects are still displayed in Jake but the user may not work on them.}
\def\ucpreconditions{Jake is running (GEN-01) and the project is open, active (= not paused) and selected.}
\def\uctriggers{User action (active)}
\def\ucprimaryscenario{
\begin{enumerate}
\item The user chooses to pause the project.
\item The project is paused.
\end{enumerate}
}
\def\ucalternativepath{-}
\def\ucpostconditions{The project is paused (not active).}
\def\ucexceptions{-}
\def\ucfrequency{Weekly/Sometimes}
\ucprint

\ucreset
\def\ucid{PRJ-06}
\def\ucname{Resume project}
\def\ucrationale{The user wants to resume a project that has been paused in order to continue working on it.}
\def\ucsummary{The user may resume a paused project. Resuming a paused project allows the user to work on it again.}
\def\ucpreconditions{Jake is running (GEN-01) and the project to be resumed is open, selected and has previously been paused (PRJ-05)}
\def\uctriggers{User action (active)}
\def\ucprimaryscenario{
\begin{enumerate}
\item The user chooses to resume the project.
\item The project is resumed.
\end{enumerate}
}
\def\ucalternativepath{-}
\def\ucpostconditions{The project is active again and the user can once again work with it.}
\def\ucexceptions{-}
\def\ucfrequency{Weekly/Sometimes}
\ucprint

\ucreset
\def\ucid{PRJ-07}
\def\ucname{Delete project}
\def\ucrationale{The user wants to delete a project, e.g. because they are sure they will not need it anymore.}
\def\ucsummary{The user may delete a project. By deleting a project, its project folder is deleted.}
\def\ucpreconditions{Jake is running (GEN-01) and the project to be deleted is open and selected}
\def\uctriggers{User action (active)}
\def\ucprimaryscenario{
\begin{enumerate}
\item The user chooses to delete the project.
\item The user is made aware of the lack of undoability inherent in this action and confirmation is requested.
\item The project is removed from Jake.
\item The folder is deleted from the filesystem or moved to the OS trash.
\end{enumerate}
}
\def\ucalternativepath{-}
\def\ucpostconditions{The project no longer exists within Jake and the local folder has been deleted.}
\def\ucexceptions{If there is a problem deleting the folder in the filesystem (e.g. locked files), the user is informed of this and restarting the computer and manually deleting the folder is suggested. The project is nonetheless removed from Jake.}
\def\ucfrequency{Monthly/Sometimes}
\ucprint

\ucreset
\def\ucid{PRJ-08}
\def\ucname{Search inside project}
\def\ucrationale{In larger projects, single files/folders/notes and users may be hard to find due to the large amount of items in the project. The user wants to find one or more items based on certain criteria.}
\def\ucsummary{The user may perform a search inside the project. All elements (files, folders, users, notes, tags, etc.) that match the search term are displayed in a results section.}
\def\ucpreconditions{Jake is running (GEN-01) and a project is open, active and selected}
\def\uctriggers{User action (active)}
\def\ucprimaryscenario{
\begin{enumerate}
\item The user enters search terms in a search element.
\item Jake presents the results matching the query.
\end{enumerate}
}
\def\ucalternativepath{-}
\def\ucpostconditions{The user is presented with the results of the query and can then perform the usual actions on them.}
\def\ucexceptions{-}
\def\ucfrequency{Very frequently}
\ucprint

\section{People Management}

\ucreset
\def\ucid{PPL-01}
\def\ucname{Invite user to project (''Add user'')}
\def\ucrationale{The user wants to invite another user to participate in the current project}
\def\ucsummary{A project member may invite other users to a specific project. The invited user is identified by their user id.}
\def\ucpreconditions{The user is logged in (NET-01) and a project is open, active and selected}
\def\uctriggers{User action (active)}
\def\ucprimaryscenario{
\begin{enumerate}
\item The user chooses to invite another user.
\item The user is asked to enter the Jabber ID of the user to be invited.
\item An invitation is sent to the invitee.
\end{enumerate}
}
\def\ucalternativepath{-}
\def\ucpostconditions{An invitation has been sent to the user in question.}
\def\ucexceptions{If the user entered does not exist on the server or the server for this Jabber ID cannot be contacted, the user is informed of this and given the option to resume at step 2.}
\def\ucfrequency{Weekly/Often}
\ucprint

\ucreset
\def\ucid{PPL-02}
\def\ucname{Explicitly trust project member}
\def\ucrationale{The user wants to receive files from another project member.}
\def\ucsummary{In order to receive files from another project member, the user must explicitly grant trust to this specific project member.}
\def\ucpreconditions{Jake is running (GEN-01), a project is open, active and selected and the project member to be trusted exists in the context of the project.}
\def\uctriggers{User action (active)}
\def\ucprimaryscenario{
\begin{enumerate}
\item The user chooses to explicitly trust another member.
\item The member is identified as trusted.
\end{enumerate}
}
\def\ucalternativepath{-}
\def\ucpostconditions{The trustee is now trusted and the user can receive files from them.}
\def\ucexceptions{-}
\def\ucfrequency{Weekly/Often}
\ucprint

\ucreset
\def\ucid{PPL-03}
\def\ucname{Revoke trust (''Remove user'')}
\def\ucrationale{The user wants to revoke trust granted to a specific project member (e.g. because they share malicious files)}
\def\ucsummary{The user may revoke the explicit trust granted to a specific project member.}
\def\ucpreconditions{Jake is running (GEN-01), a project is open, active and selected and the project member for which trust should be revoked exists in the context of the project.}
\def\uctriggers{User action (active)}
\def\ucprimaryscenario{
\begin{enumerate}
\item The user chooses to revoke trust from a project member.
\item The user is asked to confirm this action.
\item Trust is revoked from the chosen member.
\end{enumerate}
}
\def\ucalternativepath{-}
\def\ucpostconditions{The project member in question is no longer trusted.}
\def\ucexceptions{-}
\def\ucfrequency{Monthly/Sometimes}
\ucprint

\ucreset
\def\ucid{PPL-04}
\def\ucname{Set auto add/remove}
\def\ucrationale{The user wants to have changes to the project member list of certain other members propagated to their own}
\def\ucsummary{The user may set/remove the auto add/remove flag on project members on his/her project member list.}
\def\ucpreconditions{Jake is running (GEN-01), a project is open, active and selected and the specified user exists in the context of the project}
\def\uctriggers{User action (active)}
\def\ucprimaryscenario{
\begin{enumerate}
\item The user chooses to set auto add/remove on a project member.
\item The flag is set for this project member.
\end{enumerate}
}
\def\ucalternativepath{-}
\def\ucpostconditions{Peers from the specified user will now always be added to the user's own member list (and removed, see SRS).}
\def\ucexceptions{-}
\def\ucfrequency{Weekly/Sometimes (e.g. when a new user is added)}
\ucprint

\ucreset
\def\ucid{PPL-05}
\def\ucname{Show project member list}
\def\ucrationale{The user wants to see all members of the current project}
\def\ucsummary{Display a list of project members.}
\def\ucpreconditions{Jake is running (GEN-01) and a project is open, active and selected}
\def\uctriggers{User action (active)}
\def\ucprimaryscenario{
\begin{enumerate}
\item The user chooses to display the project member list.
\item The project member list is displayed.
\end{enumerate}
}
\def\ucalternativepath{-}
\def\ucpostconditions{The project member list is displayed.}
\def\ucexceptions{-}
\def\ucfrequency{Daily/Often}
\ucprint

\ucreset
\def\ucid{PPL-07}
\def\ucname{Receive member alert}
\def\ucrationale{The user wants to know who is added to or removed from the project by other members}
\def\ucsummary{If a user is added/removed from the project by another project member, the user will be notified.}
\def\ucpreconditions{The user is logged in (NET-01) and the project in question is open and active}
\def\uctriggers{Incoming member alert (passive)}
\def\ucprimaryscenario{
\begin{enumerate}
\item A member alert is received by Jake.
\item The user is notified of the new/removed project member.
\end{enumerate}
}
\def\ucalternativepath{-}
\def\ucpostconditions{The user has been made aware of the new/removed project member}
\def\ucexceptions{-}
\def\ucfrequency{Sometimes/weekly}
\ucprint

\section{File Management - Local/Offline Operations}

\ucreset
\def\ucid{FIM-01}
\def\ucname{Show files}
\def\ucrationale{The user wants to see the files that are part of a project}
\def\ucsummary{The files in the project are displayed}
\def\ucpreconditions{Jake is running (GEN-01) and a project is open, active and selected}
\def\uctriggers{User action (active)}
\def\ucprimaryscenario{
\begin{enumerate}
\item The user chooses to display the files of a project.
\item The files are displayed.
\end{enumerate}
}
\def\ucalternativepath{-}
\def\ucpostconditions{The files, if any, are displayed.}
\def\ucexceptions{-}
\def\ucfrequency{Very often}
\ucprint

\ucreset
\def\ucid{FIM-02}
\def\ucname{Import files}
\def\ucrationale{The user wants to add further files into the project folder}
\def\ucsummary{The user specifies files/folder to be imported into Jake. The rename assistant may be awakened by this action.}
\def\ucpreconditions{Jake is running (GEN-01) and a project is open, active and selected}
\def\uctriggers{User action (active)}
\def\ucprimaryscenario{
\begin{enumerate}
\item The user chooses to add further files.
\item The user is prompted to select the desired files.
\item The files are added to the project and copied to the project folder.
\end{enumerate}
}
\def\ucalternativepath{The user may just add the files by means of copying them into the project folder in the local filesystem.\\ If a file doesn't conform to the Jake naming standards, the user will be prompted if they want to rename the file (see FIM-11)}
\def\ucpostconditions{The files are now part of the project}
\def\ucexceptions{-}
\def\ucfrequency{Daily/Often}
\ucprint

\ucreset
\def\ucid{FIM-03}
\def\ucname{Reflect changes}
\def\ucrationale{Changes in the project folder should be mirrored within Jake}
\def\ucsummary{Whenever changes in the project folder occur, Jake reflects it to the user.}
\def\ucpreconditions{Jake is running (GEN-01) and the project where the change occurred is open and active}
\def\uctriggers{A change occurs in the filesystem (passive)}
\def\ucprimaryscenario{
\begin{enumerate}
\item New files are added to the project view and files that no longer exist are removed from it. Files that have changed are updated accordingly.
\end{enumerate}
}
\def\ucalternativepath{-}
\def\ucpostconditions{The change is reflected within the project view}
\def\ucexceptions{If a newly created file doesn't conform to the Jake naming standards, the user will be prompted if they want to rename the file (see FIM-11)}
\def\ucfrequency{Very often}
\ucprint

\ucreset
\def\ucid{FIM-04}
\def\ucname{Delete files/folders}
\def\ucrationale{The user wants to delete one or more files/folder from within Jake}
\def\ucsummary{Files/folders are deleted after requesting confirmation.}
\def\ucpreconditions{Jake is running (GEN-01) and a project is open, active and selected and contains at least one file or folder}
\def\uctriggers{User action (active)}
\def\ucprimaryscenario{
\begin{enumerate}
\item The user selects one or more files/folders to be deleted.
\item The user chooses to delete these items.
\item The user is prompted for confirmation.
\item The items are deleted or moved to the OS trash.
\end{enumerate}
}
\def\ucalternativepath{If one or more of the selected files are under a soft lock, the user is informed of this and prompted if they really want to proceed..}
\def\ucpostconditions{The files are no longer part of the project and are moved to the OS trash}
\def\ucexceptions{If the files can't be deleted for some reason (e.g. access violation because they are still in use by some other program), the user is informed of this and it may be suggested that they close the program and/or try to restart their computer and manually delete the files.}
\def\ucfrequency{Daily/Often}
\ucprint

\ucreset
\def\ucid{FIM-05}
\def\ucname{Rename file/folder}
\def\ucrationale{The user wants to rename a file or folder from within Jake.}
\def\ucsummary{Rename a file or a folder.}
\def\ucpreconditions{Jake is running (GEN-01) and a project is open, active and selected and contains at least one file or folder}
\def\uctriggers{User action (active)}
\def\ucprimaryscenario{
\begin{enumerate}
\item The user selects a file/folder to be renamed.
\item The user chooses to rename this item.
\item The user is prompted to enter a new name for the item in question.
\item The item is renamed.
\end{enumerate}
}
\def\ucalternativepath{If the selected file is under a soft lock, the user is informed of this and prompted if they really want to proceed.}
\def\ucpostconditions{The item is renamed both within the project and in the local file system}
\def\ucexceptions{If there is an error renaming the file (e.g. a file with the same name already exists in the same folder), the user is informed of this may resume the process at step 3.}
\def\ucfrequency{Daily/Often}
\ucprint

\ucreset
\def\ucid{FIM-06}
\def\ucname{Move files/folders}
\def\ucrationale{The user wants to move files or folders within the project directory using Jake}
\def\ucsummary{Move files or folders within the project folder.}
\def\ucpreconditions{Jake is running (GEN-01) and a project is open, active and selected and contains at least one file or folder as well as a suitable move target.}
\def\uctriggers{User action (active)}
\def\ucprimaryscenario{
\begin{enumerate}
\item The user selects one or more files or folders to be moved.
\item The user indicates where they want to move these items.
\item The items are moved, the project view is updated and the changes are applied to the local file system.
\end{enumerate}
}
\def\ucalternativepath{If one or more of the files in question are under a soft lock, the user is informed of this and prompted if they really want to proceed.}
\def\ucpostconditions{The items are in their respective new location both within the project and on the filesystem.}
\def\ucexceptions{If the operation is invalid in some way (e.g. file access violation), the user is informed of this and no changes are applied. If items of the same name already exist in the target folder, the user is informed of this and given the option to either overwrite all or none of these files.}
\def\ucfrequency{Often/Daily}
\ucprint

\ucreset
\def\ucid{FIM-07}
\def\ucname{Create folder}
\def\ucrationale{The user wants to create a new folder inside a project}
\def\ucsummary{Create a new folder from within Jake}
\def\ucpreconditions{Jake is running (GEN-01) and a project is open, active and selected}
\def\uctriggers{User action (active)}
\def\ucprimaryscenario{
\begin{enumerate}
\item The user selects a parent folder for the new folder.
\item The user chooses to create a new folder at the specified location.
\item The new folder is created.
\end{enumerate}
}
\def\ucalternativepath{-}
\def\ucpostconditions{The new folder is created in the filesystem and displayed in the project.}
\def\ucexceptions{If the location or folder name are invalid, the user is informed of this and no changes are applied.}
\def\ucfrequency{Often/Daily}
\ucprint

\ucreset
\def\ucid{FIM-08}
\def\ucname{Launch file}
\def\ucrationale{The user wants to launch a file and display its contents}
\def\ucsummary{Launch a file with its associated external application from within Jake.}
\def\ucpreconditions{Jake is running (GEN-01) and a project is open, active and selected and contains at least one file}
\def\uctriggers{User action (active)}
\def\ucprimaryscenario{
\begin{enumerate}
\item The user selects a file.
\item The user chooses to launch the selected file.
\item The file is launched by means of the OS.
\end{enumerate}
}
\def\ucalternativepath{If the selected file is under a soft lock, the user is informed of this and prompted if they really want to proceed.}
\def\ucpostconditions{The file is opened in the external application.}
\def\ucexceptions{If the file cannot be launched for some reason (e.g. access violation, the user is informed accordingly.)}
\def\ucfrequency{Very often}
\ucprint

\ucreset
\def\ucid{FIM-09}
\def\ucname{Launch folder in OS-specific file browser}
\def\ucrationale{The user wants to open a folder in the OS's file browser}
\def\ucsummary{Open a folder in the os specific file browser.}
\def\ucpreconditions{Jake is running (GEN-01) and a project is open, active and selected and contains at least one folder}
\def\uctriggers{User action (active)}
\def\ucprimaryscenario{
\begin{enumerate}
\item The user selects a folder.
\item The user chooses to launch the selected folder.
\item The folder is launched in the OS file browser.
\end{enumerate}
}
\def\ucalternativepath{-}
\def\ucpostconditions{The folder is open in the OS file browser}
\def\ucexceptions{If the OS file browser cannot be launched, the user is informed accordingly.}
\def\ucfrequency{Very often}
\ucprint

\ucreset
\def\ucid{FIM-10}
\def\ucname{Run rename assistant}
\def\ucrationale{The user wants to share files that have illegal filenames}
\def\ucsummary{Run the rename assistant to rename files/folders with illegal name.}
\def\ucpreconditions{Jake is running (GEN-01), a project is open, active and selected and one or more files with illegal names exist within it}
\def\uctriggers{User action (active)}
\def\ucprimaryscenario{
\begin{enumerate}
\item The user chooses to have their illegally-named files renamed so they can be shared.
\item The user is prompted whether they really want to perform this operation and is shown a list of files that would be affected.
\item The files are renamed.
\end{enumerate}
}
\def\ucalternativepath{-}
\def\ucpostconditions{The files with illegal filenames have been renamed.}
\def\ucexceptions{If one or more files could not be renamed (e.g. access violation or a file with the same "clean" name exists already), the user is informed of this and it is suggested that they fix the problem manually.}
\def\ucfrequency{Sometimes/Weekly}
\ucprint

\ucreset
\def\ucid{FIM-11}
\def\ucname{Offer renaming}
\def\ucrationale{The user wants to share files that are currently unable to be shared because of an illegal filename.}
\def\ucsummary{The rename assistant is implicitly started when files with illegal filenames are added to the project folder.}
\def\ucpreconditions{Jake is running (GEN-01), a project is open and active and one or more files with illegal names exist within}
\def\uctriggers{On import of illegally-named files (passive), triggered by FIM-02.}
\def\ucprimaryscenario{
\begin{enumerate}
\item The user is informed about the existence of illegally named files, including a list of affected files, and is asked whether they want to rename them right now.
\item The files are renamed.
\end{enumerate}
}
\def\ucalternativepath{If the user rejects the renaming prompt, the files continue to exist both within the project and on the local filesystem. The user can then manually or automatically rename them at a later date.}
\def\ucpostconditions{The files with illegal filenames have been renamed.}
\def\ucexceptions{If one or more files could not be renamed (e.g. access violation or a file with the same "clean" name exists already), the user is informed of this and it is suggested that they fix the problem manually.}
\def\ucfrequency{Sometimes/weekly}
\ucprint

\ucreset
\def\ucid{FIM-12}
\def\ucname{Add tag to file}
\def\ucrationale{The user wants to add a tag to a file.}
\def\ucsummary{Add a tag to a file (one file and one tag at a time).}
\def\ucpreconditions{Jake is running (GEN-01) and a project is open, active and selected and contains at least one file}
\def\uctriggers{User action (active)}
\def\ucprimaryscenario{
\begin{enumerate}
\item The user selects a file.
\item The user chooses to add a tag.
\item The user is prompted to enter a tag name.
\item The tag is added to the file.
\item The tag change is announced to other project members.
\end{enumerate}
}
\def\ucalternativepath{If the file is already tagged with a tag of the same name, the new tag is ignored.}
\def\ucpostconditions{The file is tagged with a tag of the chosen name and this change is announced}
\def\ucexceptions{-}
\def\ucfrequency{Very Often}
\ucprint

\ucreset
\def\ucid{FIM-13}
\def\ucname{Remove tag from file}
\def\ucrationale{The user wants to remove a tag from a file.}
\def\ucsummary{Remove a tag from a file. (one tag from one file at a time).}
\def\ucpreconditions{Jake is running (GEN-01), a project is open, active and selected and a tagged file exists within}
\def\uctriggers{User action (active)}
\def\ucprimaryscenario{
\begin{enumerate}
\item The user selects a tagged file.
\item The user chooses to remove a tag from the file.~ 
\item The tag is removed from the file.
\item The tag change is announced to other project members.
\end{enumerate}
}
\def\ucalternativepath{-}
\def\ucpostconditions{The file is no longer tagged with a tag of that name and this change is announced}
\def\ucexceptions{-}
\def\ucfrequency{Very Often}
\ucprint

\ucreset
\def\ucid{FIM-14}
\def\ucname{Show file properties}
\def\ucrationale{The user wants to get more information about a file in the project}
\def\ucsummary{Show detailed file properties, i.e name, size, last edit, status, etc.}
\def\ucpreconditions{Jake is running (GEN-01) and a project is open, active, selected and contains at least one file}
\def\uctriggers{User action (active)}
\def\ucprimaryscenario{
\begin{enumerate}
\item The user selects a file.
\item The user chooses to display further information.
\item Further information about the file is shown.
\end{enumerate}
}
\def\ucalternativepath{-}
\def\ucpostconditions{The user received detailed information about the selected file}
\def\ucexceptions{-}
\def\ucfrequency{Sometimes}
\ucprint

\section{File Management - Online/Sharing Operations}

\ucreset
\def\ucid{FIM-15}
\def\ucname{Announce file}
\def\ucrationale{The user wants to announce a file to the rest of the project members}
\def\ucsummary{Announce a file and specify a announce message.}
\def\ucpreconditions{Jake is running (GEN-01) and a project is open, active and selected and contains at least one file that has been changed (added, modified, moved, deleted,...)}
\def\uctriggers{User action (active) or locally modified file with auto-announce enabled (passive)}
\def\ucprimaryscenario{
\begin{enumerate}
\item The user selects a file.
\item The user chooses to announce the selected file.
\item The file is announced.
\end{enumerate}
}
\def\ucalternativepath{The file is automatically announced should the auto-announce feature be enabled.}
\def\ucpostconditions{The file is announced and a suitable log entry has been created.}
\def\ucexceptions{-}
\def\ucfrequency{Very often}
\ucprint

\ucreset
\def\ucid{FIM-16}
\def\ucname{Download file}
\def\ucrationale{The user wants to download a file from a remote peer}
\def\ucsummary{Download a file from a remote peer}
\def\ucpreconditions{The user is logged in (NET-01) and a project is open, active and selected that has a newer remote version of a file available}
\def\uctriggers{User action (active) or remotely modified file with auto-pull enabled (passive)}
\def\ucprimaryscenario{
\begin{enumerate}
\item The user selects a file.
\item The user chooses to download the selected file.
\item The file is downloaded.
\end{enumerate}
}
\def\ucalternativepath{The file is automatically downloaded should the auto-pull feature be enabled.}
\def\ucpostconditions{The file exists locally (both in project \& filesystem)}
\def\ucexceptions{-}
\def\ucfrequency{Very often}
\ucprint

\ucreset
\def\ucid{FIM-17}
\def\ucname{Resolve conflict}
\def\ucrationale{When a conflict occurs, the user wants to receive simple assistance in resolving it.}
\def\ucsummary{Run the conflict resolution assistant.}
\def\ucpreconditions{The user is logged in (NET-01) and a project is open and active and an announce has been performed}
\def\uctriggers{A local announce and a remote change of file status have resulted in a conflict situation (passive) or user action (active, choosing to resolve an earlier conflict)}
\def\ucprimaryscenario{
\begin{enumerate}
\item The user is informed of the conflict and prompted for action to resolve it (and can optionally view, save and edit/merge the two conflicting items).
\item The user chooses a way of resolving the conflict (overwrite remote, overwrite local).
\item The selected action is performed.
\end{enumerate}
}
\def\ucalternativepath{The user can choose to temporarily dismiss the message and resolve the conflict later by some means that results in step 1 in this use case being triggered.}
\def\ucpostconditions{There is no longer an acute conflict or the user is aware of actions that need to be taken.}
\def\ucexceptions{A new conflict may occur when announcing a merged file and the user is informed of this should it happen.}
\def\ucfrequency{Sometimes}
\ucprint

\ucreset
\def\ucid{FIM-18}
\def\ucname{Set soft lock}
\def\ucrationale{The user wants to edit a file alone, trying to make sure that nobody else edits it at the same time}
\def\ucsummary{Set the softlock for a file, optionally specifying a locking message.}
\def\ucpreconditions{Jake is running (GEN-01) and a project is open, active and selected and contains at least one file}
\def\uctriggers{User action (active)}
\def\ucprimaryscenario{
\begin{enumerate}
\item The user selects a file.
\item The user chooses to set the file's soft lock, optionally specifying a message to indicate their reason for setting the lock.
\end{enumerate}
}
\def\ucalternativepath{-}
\def\ucpostconditions{The file is locked.}
\def\ucexceptions{-}
\def\ucfrequency{Sometimes}
\ucprint

\ucreset
\def\ucid{FIM-19}
\def\ucname{Unset soft lock}
\def\ucrationale{The user has finished editing a file that they have locked and wants to enable other project members to work on it again.}
\def\ucsummary{Unset the soft lock for a file.}
\def\ucpreconditions{Jake is running (GEN-01) and a project is open, active and selected and contains at least one locked file}
\def\uctriggers{User action (active)}
\def\ucprimaryscenario{
\begin{enumerate}
\item The user selects a locked file.
\item The user chooses to unset the file's soft lock.
\item The file is unlocked.
\end{enumerate}
}
\def\ucalternativepath{-}
\def\ucpostconditions{The file is no longer locked.}
\def\ucexceptions{-}
\def\ucfrequency{Sometimes}
\ucprint

\section{Note Management - Local/Offline Operations}

\ucreset
\def\ucid{NTE-01}
\def\ucname{Show notes}
\def\ucrationale{The user wants to display notes associated with a project}
\def\ucsummary{Notes can be read from within Jake}
\def\ucpreconditions{Jake is running (GEN-01) and a project is open, active and selected}
\def\uctriggers{User action (active)}
\def\ucprimaryscenario{
\begin{enumerate}
\item The user chooses to display the notes of a project.
\item The notes are displayed.
\end{enumerate}}
\def\ucalternativepath{If no notes exist (yet), this should be made obvious to the user.}
\def\ucpostconditions{The user can browse through the notes.}
\def\ucexceptions{-}
\def\ucfrequency{Very often}
\ucprint

\ucreset
\def\ucid{NTE-02}
\def\ucname{Add note}
\def\ucrationale{The user wants to add a new note}
\def\ucsummary{Add a note by specifying its content}
\def\ucpreconditions{Jake is running (GEN-01) and a project is open, active and selected}
\def\uctriggers{User action (active)}
\def\ucprimaryscenario{
\begin{enumerate}
\item The user chooses to add a new note.
\item The user is prompted for the note content.
\item The note is added.
\end{enumerate}}
\def\ucalternativepath{-}
\def\ucpostconditions{The new note is part of the project}
\def\ucexceptions{If the note content is empty, the user is informed of this and given the option to resume at step 2 in the process.}
\def\ucfrequency{Often}
\ucprint

\ucreset
\def\ucid{NTE-03}
\def\ucname{Delete note}
\def\ucrationale{The user wants to delete an unneeded note.}
\def\ucsummary{Delete note(s)}
\def\ucpreconditions{Jake is running (GEN-01) and a project is open, active and selected and contains at least one note.}
\def\uctriggers{User action (active)}
\def\ucprimaryscenario{
\begin{enumerate}
\item The user selects one or more notes to be deleted.
\item The user chooses to delete these notes.
\item The user is prompted for confirmation and informed that this action cannot be undone.
\item The notes are deleted.
\end{enumerate}}
\def\ucalternativepath{-}
\def\ucpostconditions{The deleted notes no longer exist within the project}
\def\ucexceptions{-}
\def\ucfrequency{Often}
\ucprint

\ucreset
\def\ucid{NTE-04}
\def\ucname{Edit note}
\def\ucrationale{The user wants to edit an existing note to update its contents}
\def\ucsummary{Editing a note to change its content.}
\def\ucpreconditions{Jake is running (GEN-01) and a project is open, active and selected and contains at least one note.}
\def\uctriggers{User action(active)}
\def\ucprimaryscenario{
\begin{enumerate}
\item The user selects a note to be edited.
\item The user is prompted for the notes new content (starting off with its old content).
\item The note is updated.
\end{enumerate}}
\def\ucalternativepath{-}
\def\ucpostconditions{The note now contains the specified new content and its new ''title'' (first line) is displayed in the note view}
\def\ucexceptions{-}
\def\ucfrequency{Often}
\ucprint

\ucreset
\def\ucid{NTE-05}
\def\ucname{Add tag to note}
\def\ucrationale{The user wants to add a tag to a note.}
\def\ucsummary{The user may add a tag to a note (one note at a time).}
\def\ucpreconditions{Jake is running (GEN-01) and a project is open, active and selected}
\def\uctriggers{User action (active)}
\def\ucprimaryscenario{
\begin{enumerate}
\item The user selects a note.
\item The user chooses to add a tag.
\item The user is prompted to enter a tag name.
\item The tag is added to the note.
\item The tag change is announced to other project members.
\end{enumerate}
}
\def\ucalternativepath{If the note is already tagged with a tag of the same name, the new tag is ignored.}
\def\ucpostconditions{The note is tagged with a tag of the chosen name and this change is announced}
\def\ucexceptions{-}
\def\ucfrequency{Very Often}
\ucprint

\ucreset
\def\ucid{NTE-06}
\def\ucname{Remove tag from note}
\def\ucrationale{The user wants to remove a tag from a note.}
\def\ucsummary{The user may remove a tag from a note. (one tag from one note at a time)}
\def\ucpreconditions{Jake is running (GEN-01), a project is open, active and selected and a tagged note exists within}
\def\uctriggers{User action (active)}
\def\ucprimaryscenario{
\begin{enumerate}
\item The user selects a tagged note.
\item The user chooses to remove a tag from the note.
\item The tag is removed from the note.
\item The tag change is announced to other project members.
\end{enumerate}
}
\def\ucalternativepath{-}
\def\ucpostconditions{The note is no longer tagged with a tag of that name and this change is announced}
\def\ucexceptions{-}
\def\ucfrequency{Very Often}
\ucprint

\section{Note Management - Online/Sharing Operations}

\ucreset
\def\ucid{NTE-07}
\def\ucname{Announce note}
\def\ucrationale{The user wants to announce a note to the rest of the project members}
\def\ucsummary{Announce a note and specify an announce message.}
\def\ucpreconditions{Jake is running (GEN-01) and a project is open, active and selected and contains at least one note that has not yet been announced}
\def\uctriggers{User action (active) or locally modified note with auto-announce enabled (passive)}
\def\ucprimaryscenario{
\begin{enumerate}
\item The user selects a note.
\item The user chooses to announce the selected note.
\item The note is announced.
\end{enumerate}
}
\def\ucalternativepath{The note is automatically announced should the auto-announce feature be enabled.}
\def\ucpostconditions{The note is announced and a suitable log entry has been created.}
\def\ucexceptions{-}
\def\ucfrequency{Very often}
\ucprint

\ucreset
\def\ucid{NTE-08}
\def\ucname{Download note}
\def\ucrationale{The user wants to download a note from a remote peer}
\def\ucsummary{Download a note from a remote peer}
\def\ucpreconditions{The user is logged in (NET-01) and a project is open, active and selected that has a newer remote version of a note available}
\def\uctriggers{User action (active) or remotely modified note with auto-pull enabled (passive)}
\def\ucprimaryscenario{
\begin{enumerate}
\item The user selects a note (stub).
\item The user chooses to download the selected note.
\item The note is downloaded.
\end{enumerate}
}
\def\ucalternativepath{The note is automatically downloaded should the auto-pull feature be enabled.}
\def\ucpostconditions{The note is part of the local project}
\def\ucexceptions{-}
\def\ucfrequency{Very often}
\ucprint

\ucreset
\def\ucid{NTE-09}
\def\ucname{Resolve note conflict}
\def\ucrationale{When a conflict occurs, the user wants to receive simple assistance in resolving it.}
\def\ucsummary{Run the conflict resolution assistant.}
\def\ucpreconditions{The user is logged in (NET-01) and a project is open and active and an announce has been performed on a note}
\def\uctriggers{A local announce and a remote change of note status have resulted in a conflict situation (passive) or user action (active, choosing to resolve an earlier conflict)}
\def\ucprimaryscenario{
\begin{enumerate}
\item The user is informed of the conflict and prompted for action to resolve it.
\item The user chooses a way of resolving the conflict (overwrite remote, overwrite local).
\item The selected action is performed.
\end{enumerate}
}
\def\ucalternativepath{The user can choose to temporarily dismiss the message and resolve the conflict later by some means that results in step 1 in this use case being triggered.}
\def\ucpostconditions{There is no longer an acute conflict or the user is aware of actions that need to be taken.}
\def\ucexceptions{A new conflict may occur when announcing the chosen and the user is informed of this should it happen.}
\def\ucfrequency{Sometimes}
\ucprint

\ucreset
\def\ucid{NTE-10}
\def\ucname{Set soft lock (note)}
\def\ucrationale{The user wants to edit a note alone, trying to make sure that nobody else edits it at the same time}
\def\ucsummary{Set the softlock for a note, optionally specifying a locking message.}
\def\ucpreconditions{Jake is running (GEN-01) and a project is open, active and selected and contains at least one note}
\def\uctriggers{User action (active)}
\def\ucprimaryscenario{
\begin{enumerate}
\item The user selects a note.
\item The user chooses to set the note's soft lock, optionally specifying a message to indicate their reason for setting the lock.
\end{enumerate}
}
\def\ucalternativepath{-}
\def\ucpostconditions{The note is locked}
\def\ucexceptions{-}
\def\ucfrequency{Sometimes}
\ucprint

\ucreset
\def\ucid{NTE-11}
\def\ucname{Unset soft lock (note)}
\def\ucrationale{The user has finished editing a note that they have locked and wants to enable other project members to work on it again.}
\def\ucsummary{Unset the soft lock for a note.}
\def\ucpreconditions{Jake is running (GEN-01) and a project is open, active and selected and contains at least one locked note}
\def\uctriggers{User action (active)}
\def\ucprimaryscenario{
\begin{enumerate}
\item The user selects a locked note.
\item The user chooses to unset the note's soft lock.
\item The note is unlocked.
\end{enumerate}
}
\def\ucalternativepath{-}
\def\ucpostconditions{The note is no longer locked}
\def\ucexceptions{-}
\def\ucfrequency{Sometimes}
\ucprint

\section{Log Operations}

\ucreset
\def\ucid{LOG-01}
\def\ucname{Display global project log}
\def\ucrationale{The user wants to see all recent changes in the project}
\def\ucsummary{A list of recent changes is available for the user to browse through}
\def\ucpreconditions{Jake is running (GEN-01) and a project is open, active and selected}
\def\uctriggers{User action (active)}
\def\ucprimaryscenario{
\begin{enumerate}
\item The user chooses to display the project log.
\item The project log is displayed.
\end{enumerate}
}
\def\ucalternativepath{If there hasn't (yet) been any activity in the project, the user is informed of this.}
\def\ucpostconditions{The user can view the project log}
\def\ucexceptions{-}
\def\ucfrequency{Sometimes}
\ucprint

\ucreset
\def\ucid{LOG-02}
\def\ucname{Display file log}
\def\ucrationale{The user wants to see all recent changes of a file}
\def\ucsummary{Display the log messages associated with a specific file}
\def\ucpreconditions{Jake is running (GEN-01) and a project is open, active and selected and contains at least one file}
\def\uctriggers{User action (active)}
\def\ucprimaryscenario{
\begin{enumerate}
\item The user selects a file.
\item The user chooses to display the log of this file.
\item The file log is displayed.
\end{enumerate}}
\def\ucalternativepath{-}
\def\ucpostconditions{The user can view the file log}
\def\ucexceptions{-}
\def\ucfrequency{Sometimes}
\ucprint

\ucreset
\def\ucid{LOG-03}
\def\ucname{Display note log}
\def\ucrationale{The user wants to see all recent changes of a note}
\def\ucsummary{Display the log messages associated with a specific note}
\def\ucpreconditions{Jake is running (GEN-01) and a project is open, active and selected and contains at least one note}
\def\uctriggers{User action (active)}
\def\ucprimaryscenario{
\begin{enumerate}
\item The user selects a note.
\item The user chooses to display the log of this note.
\item The note log is displayed.
\end{enumerate}}
\def\ucalternativepath{-}
\def\ucpostconditions{The user can view the note log}
\def\ucexceptions{-}
\def\ucfrequency{Sometimes}
\ucprint

\section{Preferences}

\ucreset
\def\ucid{PRF-01}
\def\ucname{Edit project preferences}
\def\ucrationale{The user wants to change preferences of the project}
\def\ucsummary{Project preferences, such as auto-pull/-announce, can be changed by the user}
\def\ucpreconditions{Jake is running (GEN-01) and a project is open, active and selected}
\def\uctriggers{User action (active)}
\def\ucprimaryscenario{
\begin{enumerate}
\item The user chooses to edit project preferences.
\item The user is given a range of options to choose from and change.
\item If the user changes an option, the changes are applied immediately.
\end{enumerate}}
\def\ucalternativepath{Options considered ''dangerous'', should there be any, will prompt additional confirmation from the user}
\def\ucpostconditions{The preferences are saved and will be considered from this point on.}
\def\ucexceptions{If the user enters an invalid value for an option, they will be informed of this and must correct it.}
\def\ucfrequency{Sometimes}
\ucprint

\ucreset
\def\ucid{PRF-02}
\def\ucname{Edit global preferences}
\def\ucrationale{The user wants to change global preferences}
\def\ucsummary{Global preferences, such as i18n, can be changed by the user}
\def\ucpreconditions{Jake is running (GEN-01)}
\def\uctriggers{User action (active)}
\def\ucprimaryscenario{
\begin{enumerate}
\item The user chooses to edit global preferences.
\item The user is given a range of options to choose from and change.
\item If the user changes an option, the changes are applied immediately.
\end{enumerate}}
\def\ucalternativepath{Options considered ''dangerous'', should there be any, will prompt additional confirmation from the user}
\def\ucpostconditions{The preferences are saved and will be considered from this point on.}
\def\ucexceptions{If the user enters an invalid value for an option, they will be informed of this and must correct it.}
\def\ucfrequency{Sometimes}
\ucprint
